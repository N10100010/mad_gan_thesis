\section{Experiments Results}\label{body_experiments_results}
\paragraph{Motivation}\label{par_EXP_durch_motiv}
The primary motivation for investigating multi-generator GAN architecures for GDA arose from a specific suggestion by Ian Goodfellow during his appearance on the Lex Fridman Podcast \cite{fridman2019Goodfellow}. He proposed exploring the collective strength of multiple generative models trained on the same data:

\begin{quotation}
    So one thing I think is worth trying [\dots] is, what if you trained a whole lot of different generative models on the same training set, create samples from all of them and then train a classifier on that. Because each of the generative models might generalize in a slightly different way, they might capture different axes of variation, that one individual model wouldn't and then the classifier can capture all of those ideas, by training on all of their data.
\end{quotation}\citep[50:37]{fridman2019Goodfellow}

This idea, suggesting that diversity in generation could lead to richer datasets for subsequent tasks, forms the conceptual basis for the experiments conducted in this work.

\subsection{Questions to answer}




\newpage
