\section{Introduction and Motivation}\label{introduction_and_motivation}
\pagestyle{fancy}
%\setcounter{page}{1}
Generative Adverserial Netwrorks (GANs) \cite{goodfellow2014generativeadversarialnetworks} and their variants revolutionized the field of computer vision in the year of 2014, enabling advacements in multiple areas of generating data. From \textit{Text to Image Synthesis} \cite{reed2016generativeadversarialtextimage}, \textit{Image Translation} \cite{isola2018imagetoimagetranslationconditionaladversarial}, \textit{Super Resolution} \cite{ledig2017photorealisticsingleimagesuperresolution}, \textit{Image Inpainting} \cite{pathak2016contextencodersfeaturelearning}, \textit{Style Transfer} \cite{wang2023multimodalityguidedimagestyletransfer} to \textit{Data Augmentation} \cite{shorten2019survey}, GANs have been used in a variety of applications.

The idea of using GANs for Generative Data Augmentation (GDA) has already been applied sucessfully, e.g.: in computer vision \cite{Li2025comprehensivesurvedeepimages} or for creating music \cite{ji2020comprehensivesurveydeepmusic}. Especially the former survey \textit{A Comprehensive Survey of Image Generation Models Based on Deep Learning} has, along Variational Auto Encoders (VAEs), a dedicated focus on GANs.

Regardless of the achievements of GANs they suffer from multiple problems:

\begin{itemize}
    \item Failure to Converge
    \item Mode Collapse
    \item Vanishing Gradiants
    \item Unstable Gradiants
    \item Imbalance between Generator and Discriminator
\end{itemize}

% \cite{espeholt2018impala}, in the field of computer vision \cite{espeholt2019seed} and in the field of music \cite{espeholt2020music}.

\paragraph{Aim of the Thesis}\label{aim_of_the_thesis}
The aim of the thesis is to investigate the potential use....


\newpage
