\section{Related Work}\label{related_work}

%AUGMENTATION

%##########################################
% INTRO TO GDA
The effectiveness of deep learning models is intrinsically linked to the availability of large and diverse datasets for training. Models with deep and complex architectures require extensive exposure to a wide range of data to learn underlying patterns and generalize well to unseen instances. Insufficient training data can lead to a phenomena called \textit{overfitting}, where a model becomes too specialized to the training data, failing to perform accurately on previously unencountered data \cite{Ying2019overfittinganditssolutions}.

Data augmentation artificially expands the amounts and diversity of training datasets by creating modified versions of existing data or by generating entirely new instances. To mitigate the problem of data scarcity and improve generalization capabilities of deep learning models, data augmentation techniques became indispensable. \\
%##########################################

%##########################################
% TRAD TECHNIQUES FOR GDA
\noindent\textbf{Traditional Data Augmentation}\label{traditional_data_augmentation} \\
Traditional data augmentation on images typically involves applying various transformations to existing data. For image based data, augmentations can take a variety of forms such as
\footnote{More categories of traditional data augmentation techniques exists, such as Occlusion-Based, Composition-Based, Domain-Specific or Adversarial Augmentation. For the purpose of this work,  soley the afore mentioned are discussed in greater detail.}
: \textit{Geometric Augmentation}, \textit{Photometric Augmentation}, \textit{Noise-Corruption Augmentation} \ref{theoretical_tda}.

\noindent 
The sucess of the above mentioned augmentation techniques is established in many papers \cite{perez2017effectivenessdataaugmentationimage}, \cite{NIPS2012_c399862d}, \cite{Ying2019overfittinganditssolutions}, \cite{shorten2019survey}, \cite{WanLiZeiler2013}.\\



%##########################################


%##########################################
% GANS Creating Data
\noindent\textbf{Generative Data Augmentation using Deep Convolutional GANs}\label{dcgans_data_augmentation} \\
The basic GAN framework introduced by Goodfellow and colleagues offers a high degree of flexibility and can be adapted for specific augmentation tasks. It can be applied to generate music \cite{dong2017museganmultitracksequentialgenerative}, speech \cite{li2022ttsgantransformerbasedtimeseriesgenerative}, text \cite{yu2017seqgansequencegenerativeadversarial}, images \cite{goodfellow2014generativeadversarialnetworks} or other instances of data, e.g. tabular data \cite{xu2019modelingtabulardatausing}.
%##########################################

%##########################################
% GANS for GDA
Especially for image data, \textit{Deep Convolutional GANs} (DCGANs) \cite{Radford2015DCGAN} represent a significant advacement in applying GANs to image data augmentation \cite{huang2022tutorial}. Their architecture specifically utilizes \textit{Convolutional Neural Network} (CNNs) \cite{LeCun1989firstcnnpaper} in both, the generator and the discriminator. The use of CNNs allow DCGANs to learn hierarchical features from the input images effectively and capture the spatial relationship and structure inherent in the training data. This leads to the generation of more realistic and coherent synthetic images. A study from Zhau et al. \cite{zhao2023gan} applied DCGANs, along their adjusted versions on multiple dataset, including \textit{Fashion MNIST} and \textit{Cifar10}. With their experimental setup, they achieved consistent significant improvements over multiple datasets using the DCGAN-architechture, compared to their baseline. \\

% TODO:
% nice study failing to apply DCGANs to CIFAR10: https://www.researchgate.net/publication/383101057_Exploiting_Deep_Convolutional_Generative_Adversarial_Network_Generated_Images_for_Enhanced_Image_Classification/fulltext/66bce091311cbb094938deea/Exploiting-Deep-Convolutional-Generative-Adversarial-Network-Generated-Images-for-Enhanced-Image-Classification.pdf?origin=publication_detail&_tp=eyJjb250ZXh0Ijp7ImZpcnN0UGFnZSI6InB1YmxpY2F0aW9uIiwicGFnZSI6InB1YmxpY2F0aW9uRG93bmxvYWQiLCJwcmV2aW91c1BhZ2UiOiJwdWJsaWNhdGlvbiJ9fQ&__cf_chl_tk=jOsX37wVwQerStwY02d0Dn.GJqfKfJdurmBpFfvIDmA-1742762383-1.0.1.1-Y2HfBG9ppzh.6K_kc.NyRkesLnTxRtvaVk_KZ2mzJn4
%##########################################


%##########################################
% Transition to cGANs
Inherently in the vanilla version of GANs or the DCGANs realization of using convolutional layers, the generator's role is solely to learn the underlying data distribution of the training samples and produce instances of close resemblance to instances from the training data. This however results in unlabeled samples, not beneficial to expand data for a supervised classification task out of the box.
% Conditional GANS FOR GDA

\noindent\textbf{Generative Data Augmentation using Conditional GANs}\label{cgans_data_augmentation} \\
The introduction of \textit{Conditional Generative Adversarial Networks} (cGANs) \cite{mirza2014conditionalgenerativeadversarialnets} allows to condition the generative process by additional information, such as class labels or other modalities. The conditioning acts on both the generator and the discriminator, which means that both models have access to the same conditional information. The generator combines the random vector input and the conditioning information into a joint hidden representation. The discriminator, on the other hand, evaluates the created data from generator, given context of the conditioning information, i.e., the class label passed. This approach enables the generator to create data that adheres to specific inputs, like creating specific digits from the MNIST dataset \ref{used_datasets}. Multiple papers were able to utilize the advantages of cGANs to, e.g., unify class distributions for a stratified classifier training or generatively increase the number of images and augmenting the training data\cite{jeong2022gan}\cite{zhao2023gan}\cite{cGANGDA2025asurveyreview}\cite{wickramaratne2021conditional}.
%##########################################

%##########################################
% MADGANS FOR GDA ghosh2018madgan
\noindent\textbf{Generative Data Augmentation using MADGANs}\label{madgans_data_augmentation} \\
Regardless of the mentioned successes using GANs (DCGANs or cGANs) for GDA \ref{dcgans_data_augmentation} \ref{cgans_data_augmentation}, GANs in general have proven to be notoriously hard to train. \textit{Among them, mode collapse stands out as one of the most daunting ones.} \cite{durall2020combatingmodecollapsegan}, which limits the GANs ability to generate diverse samples, able to be assigned to all classes trained on. Another prominent problem with GANs is the Lack of inter-class diversity between generated samples.

MADGANs \cite{ghosh2018madgan} emphasis on diversity, achieved through its multi-agent architecture and the modified discriminator objective function, directly addresses these limitations. By encouraging multiple generators to specialize in different modes of the data distribution, MADGAN aims to generate a more comprehensive and diverse set of synthetic samples compared to traditional GANs and potentially other generative data augmentation techniques that might be susceptible to mode collapse. The ability of MADGAN to disentangle different modalities i.e., classes, as suggested by experiments involving diverse-class datasets, indicates its potential to generate augmented data that effectively covers both intra-class and inter-class variations. This comprehensive coverage is crucial for training robust image classifiers that can generalize well to a wide range of real-world scenarios.
%##########################################

\newpage
